\section{Type Inference Algorithm}
\label{sec:algorithm}

This section presents our type inference algorithm. This algorithm is
independent of the underlying pluggable typechecker: that is,
it applies equally well to any pluggable typechecker that performs flow-sensitive
local type inference.

\todo{Should we call $T_I$ an ``inferring type-checker''?  Having a name can
be handy.}

\todo{It would be nice to give a name to our approach, rather than always
  calling it ``our approach''.  How about ``iterated local type
  inference''?  Do we have any better ideas?}

Our key observation is that practical pluggible type-checkers use 
flow-sensitive type refinement, which is a type of local type inference.
\todo{We must define the above terms.}
The two key ideas behind our type inference approach are to expose the
results of the local type inference, and to iteratively call a type-checker
reusing these results.

Instead of building a type inference for each type system,
our key idea is to modify the underlying \emph{framework}
on which the pluggable typecheckers are built.  This converts any
type-checker built on the framework into an \emph{inferring type-checker}.

% Pluggable
% typecheckers use a modified version of the host type system
% that supports type qualifiers. Our approach modifies this support
% layer for pluggable typechecking in order to support inference for
% any typechecker.

A typechecker $T : P \rightarrow E$
takes a program and outputs a (possibly empty) set of
type errors.  An inferring typechecker using our modified framework,
$T_I : \langle P, A \rangle \rightarrow \langle E, A' \rangle$,
takes a program along with a set of additional type qualifiers $A$.
It outputs errors and
inferences (\ie $A'$, a new set of type qualifiers).
The errors $E$ are exactly those $T$ would output, if the
type qualifiers in $A$ had been written on $P$ by a programmer.
$P$ may or may not contain programmer-written type qualifiers.

\todo{Relate to the two ``key ideas'' above.}
\todo{Why in this order?  It's opposite what is described above.}
We describe the algorithm in two parts: first,
\Cref{sec:core-algorithm} gives the fixpoint
algorithm used to infer types for a particular program
using our modified typechecking framework.
\Cref{sec:instrument} explains the modifications to the pluggable
typechecking framework that enable type inference for a pluggable
typechecker $T$ (\ie convert it to $T_I$ as described above).

Note that in these two phases of our algorithm, estimates for
candidate type qualifiers move in different directions in the
type qualifier hierarchy.
%
In the outer loop (\cref{sec:core-algorithm}),
estimates only go down (or stay that same): that is,
type qualifiers become more refined.
%
In a single round of type inference, however, type qualifier
estimates go up in the hierarchy: for example, the estimate
for the type of a field is the least upper bound of the types
of all the expressions that are assigned into that field.

%% \todo{Here is a fact that I think we should explain to readers.
%%   Within a single type-checking run, the type estimates only go up.
%%   Between runs (that is, in \cref{alg:wpi-fixpoint}), the type estimates
%%   only go down.}


\subsection{Fixpoint Algorithm:  the outer loop}
\label{sec:core-algorithm}

% the core WPI fixpoint algorithm; that is, the outer WPI loop

\begin{algorithm}
  \DontPrintSemicolon
  \SetKwFunction{Infer}{infer}
  \SetKwProg{Fn}{def}{:}{}
  \SetKwInOut{Input}{input}
  \SetKwInOut{Output}{output}
  \Input{program $P$ and pluggable typechecker $T$}
  \Output{set of errors $E$ and set of type qualifiers $A$}
  \Fn{ \Infer{$P$, $T$} } {
%    \tcc{In each iteration, $\|prevA|$ is the specification set before running $T$, and $A$ is the set after.}
%    $\|prevA| \gets \emptyset, A \gets \emptyset$ \;
    $A \gets \emptyset$ \;
%    \tcc{\textsc{Instrument} modifies $T$ to both check the program and collect a candidate specification set.}
    $T_I \gets \textsc{EnableInference}(T)$ \;
%    $E, A \gets T_I(P, \|prevA|)$ \;
    \Repeat{$E = \emptyset \vee \|prevA| = A$}{
      % \tcc{Note that each element of $A$ is either a new type qualifier or refines some element of $\|prevA|$.}
      $\|prevA| \gets A$ \;
      $E, A \gets T_I(P, \|prevA|)$ \;
    }
    \Return $E, A$ \;
  }
  \caption{Iterated local type inference algorithm.  This is the ``outer
    loop'' of the approach, which iterates to a fixed point.
    The helper function \textsc{EnableInference} is defined by the modifications
    to the framework described in \cref{sec:instrument}.
    \todo{Why is the caption so narrow, not taking up the whole column?}
}
  \label{alg:wpi-fixpoint}
\end{algorithm}


The inference algorithm appears
in \cref{alg:wpi-fixpoint}.   It iteratively analyzes
the target program $P$ with an inferring version of the
pluggable typechecker
% whose core qualified type rules
% have been modified to support inference in the manner described in
(see \cref{sec:instrument}),
% recording intermediate results at each
% step (the sets of inferred type qualifiers $A$ and $A^{\prime}$)
until
either there are no remaining typechecking errors
(\ie $E = \emptyset$)
or the
type qualifiers reach fixpoint (\ie $\|prevA| = A$).

%% This fixpoint loop is quite general: its success for our
%% purposes depends heavily on the \textsc{Instrument} procedure
%% that outputs locally-inferred specifications. \Cref{sec:instrument}
%% describes how we implement \textsc{Instrument} in a way that is
%% applicable to any pluggable typechecker.

\begin{theorem}
\Cref{alg:wpi-fixpoint} monotonically refines types.  That is,
each element $a$ of $A$ is either 1) not present in $\|prevA|$, or
2) there exists some element $\|preva|$ in $\|prevA|$ such that $a \sqsubseteq \|preva|$
and both $\|preva|$ and $a$ qualify the same base type in the same program location.
\end{theorem}

\begin{proof}
By induction on the type rules in \cref{sec:instrument}.
\end{proof}


\subsubsection{Soundness}
\label{sec:soundness}

Any infer-then-check approach is sound so long as the ``check'' step is sound.
Even if the inference algorithm were to produce incorrect type qualifiers,
the checker would reject them.


\subsection{Modifications to the Typechecking Framework}
\label{sec:instrument}

The algorithm presented in \cref{sec:core-algorithm} works for
any pluggable typechecker that supports flow-sensitive inference:
that is, it does not require a type system implementer to write
any special rules to support type inference. Instead, this section
describes our approach to \emph{automatically} modify a given
pluggable typechecker to support inference, corresponding to the
\textsc{Enable\-Inference} helper function in \cref{alg:wpi-fixpoint}.

The key idea behind our inference approach to instrumenting the typechecker
is to modify the \emph{framework}.  Once the pluggable type-checking
framework is modified, inference is enabled for every typechecker built on it.
Our modifications
can be conceptualized at the type-qualifier-theory level: that is,
we modify the rules for typechecking \emph{any} pluggable type system
so that inference is supported,
regardless of the particular qualifiers it happens to support.

Our modifications rely on the fact that all practical pluggable type systems
do \emph{local}, intra-procedural flow-sensitive type inference.
This means that programmers rarely need to write annotations within method bodies.
Analogously, Java's \<var> keyword permits programmers to omit Java basetypes within
method bodies.
Programmers are more willing to write types on method
signatures, where they form valuable documentation.  (Our goal is to lift
even this burden, for type annotations)

\todo{Also mention
  that this seems to be a general trend in language design, citing maybe Kotlin?}

A pluggable type \emph{checker} permits programmers to leave basetypes unqualified.
On APIs (class/method/field declarations), the type checker uses defaulting
rules to assign a qualifier to each unqualified basetype.
Within a code block, the type checker performs local flow-sensitive type
inference.


\begin{figure*}
  \begin{mathpar}
    \inferrule* [right=INVOKE]
                {
                  \std{\Gamma \vdash m(f_0,\ldots,f_n) : }~\qual{q_R}~\std{\tau_R}
                  \\
                  \std{\Gamma \vdash \forall i \in 0,\ldots,n . ~f_i :~} \qual{q_{F_i}}~\std{\tau_{F_i}}
                  \\
                  \std{\Gamma \vdash \forall i \in 0,\ldots,n . ~e_i :~} \qual{q_{A_i}}~\std{\tau_{A_i}}
                  \\
                  \std{\Gamma \vdash \forall i \in 0,\ldots,n . ~} \qual{q_{A_i}}~\std{\tau_{A_i}~\sqsubseteq}~\qual{q_{F_i}}~\std{\tau_{F_i}}
                  \\
                  \infr{\infEnv \vdash \forall i \in 0,\ldots,n . ~f_i~:~q_{I_i}~\tau_{F_i}}
                }
                {
                  \std{\Gamma \vdash m(e_0,\ldots,e_n) : }~\qual{q_R}~\std{\tau_R}
                  \\
                  \infr{\infEnv \vdash \forall i \in 0,\ldots,n . ~f_i~:~\mathit{LUB_Q}(q_{A_i},~q_{I_i})~\tau_{F_i} }
                }
                
     \inferrule* [right=NEW]
                {
                  \std{\Gamma \vdash \<new T>(f_1,\ldots,f_n) : }~\qual{q_R}~\std{\tau_R}
                  \\
                  \std{\Gamma \vdash \forall i \in 1,\ldots,n . ~f_i :~} \qual{q_{F_i}}~\std{\tau_{F_i}}
                  \\
                  \std{\Gamma \vdash \forall i \in 1,\ldots,n . ~e_i :~} \qual{q_{A_i}}~\std{\tau_{A_i}}
                  \\
                  \std{\Gamma \vdash \forall i \in 1,\ldots,n . ~} \qual{q_{A_i}}~\std{\tau_{A_i}~\sqsubseteq}~\qual{q_{F_i}}~\std{\tau_{F_i}}
                  \\
                  \infr{\infEnv \vdash \forall i \in 1,\ldots,n . ~f_i~:~q_{I_i}~\tau_{F_i}}
                }
                {
                  \std{\Gamma \vdash \<new T>(e_1,\ldots,e_n) : }~\qual{q_R}~\std{\tau_R}
                  \\
                  \infr{\infEnv \vdash \forall i \in 1,\ldots,n . ~f_i~:~\mathit{LUB_Q}(q_{A_i},~q_{I_i})~\tau_{F_i} }
                }

     \inferrule* [right=FORMAL-ASSIGN]
                {
%                  \std{f~is~a~formal~parameter} \\
                  \std{\Gamma \vdash f~:}~\qual{q_F}~\std{\tau_F} \\
                  \std{\Gamma \vdash e~:}~\qual{q_A}~\std{\tau_A} \\
                  \std{\Gamma \vdash~} \qual{q_A}~\std{\tau_A~\sqsubseteq}~\qual{q_F}~\std{\tau_F} \\
                  \infr{\infEnv \vdash f~:~q_I~\tau_F}
                }
                {
                  \std{\Gamma \vdash f~:=~e} \\
                  \infr{\infEnv \vdash f~:~\mathit{LUB_Q}(q_A, q_I)~\tau_F}
                }

     \inferrule* [right=FIELD-ASSIGN]
                {
%                  \std{f~is~a~formal~parameter} \\
                  \std{\Gamma \vdash x.f~:}~\qual{q_F}~\std{\tau_F} \\
                  \std{\Gamma \vdash e~:}~\qual{q_A}~\std{\tau_A} \\
                  \std{\Gamma \vdash~} \qual{q_A}~\std{\tau_A~\sqsubseteq}~\qual{q_F}~\std{\tau_F} \\
                  \infr{\infEnv \vdash C.f~:~q_I~\tau_F}
                }
                {
                  \std{\Gamma \vdash x.f~:=~e} \\
                  \infr{\infEnv \vdash C.f~:~\mathit{LUB_Q}(q_A, q_I)~\tau_F}
                }

     \inferrule* [right=RETURN]
                {
%                  \std{f~is~a~formal~parameter} \\
                  \std{\Gamma \vdash m(f_0,\ldots,f_n)~:}~\qual{q_R}~\std{\tau_R} \\
                  \std{\Gamma \vdash e~:}~\qual{q_A}~\std{\tau_A} \\
                  \std{\Gamma \vdash~} \qual{q_A}~\std{\tau_A~\sqsubseteq}~\qual{q_R}~\std{\tau_R} \\
                  \infr{\infEnv \vdash m(f_0,\ldots,f_n)~:~q_I~\tau_R}
                }
                {
%                  \std{\Gamma \vdash \<return>~e} \\
                  \std{\<return>~e \in m} \\
                  \infr{\infEnv \vdash m(f_0,\ldots,f_n)~:~\mathit{LUB_Q}(q_A, q_I)~\tau_R}
                }

    \inferrule* [right=OVERRIDE]
                {
                  % return types
                  \std{\Gamma \vdash m_B(f_{0_B},\ldots,f_{n_B}) : }~\qual{q_{R_B}}~\std{\tau_{R_B}}
                  \\
                  \std{\Gamma \vdash m_P(f_{0_P},\ldots,f_{n_P}) : }~\qual{q_{R_P}}~\std{\tau_{R_P}}
                  \\
                  \std{\Gamma \vdash~} \qual{q_{R_B}}~\std{\tau_{R_B}~\sqsubseteq}~\qual{q_{R_P}}~\std{\tau_{R_P}}
                  \\
                  \std{\Gamma \vdash \forall i \in 0,\ldots,n_B . ~f_{B_i} :~} \qual{q_{B_i}}~\std{\tau_{B_i}}
                  \\
                  \std{\Gamma \vdash \forall i \in 0,\ldots,n_P . ~f_{P_i} :~} \qual{q_{P_i}}~\std{\tau_{P_i}}
                  \\
                  \std{\Gamma \vdash \forall i \in 0,\ldots,n_B . ~} \qual{q_{B_i}}~\std{\tau_{B_i}~\sqsubseteq}~\qual{q_{P_i}}~\std{\tau_{P_i}}
                  \\
                  \std{\vdash n_B~=~n_P}
                  \\
                  \infr{\infEnv \vdash m_1(f_{0_B},\ldots,f_{n_B})~:~q_{R_B-I}~\tau_{R_B}}
                  \\
                  \infr{\infEnv \vdash m_P(f_{0_P},\ldots,f_{n_P})~:~q_{R_P-I}~\tau_{R_P}}
                  \\
                  \infr{\infEnv \vdash \forall i \in 0,\ldots,n_B . ~f_{B_i}~:~q_{B_i-I}~\tau_{B_i}}
                  \\
                  \infr{\infEnv \vdash \forall i \in 0,\ldots,n_P .~f_{P_i}~:~q_{P_i-I}~\tau_{P_i}}
                }
                {
                  \std{\Gamma \vdash m_B(f_{0_B},\ldots,f_{n_B})~\mathit{is~a~valid~override~of}~m_P(f_{0_P},\ldots,f_{n_P})}
                  \\
                  \infr{\infEnv \vdash m_P(f_{0_P},\ldots,f_{n_P})~:~\mathit{LUB_Q}(q_{R_B-I}, q_{R_P-I})~\tau_{R_P}}
                  \\
                  \infr{\infEnv \vdash \forall i \in 0,\ldots,n_P . ~f_{P_i}~:~\mathit{LUB_Q}(q_{B_i-I},~q_{P_i-I})~\tau_{P_i} }
                }
                
  \end{mathpar}

  \todo{These rules do not seem to handle qualifier polymorphism, where the
    return type can depend on the instantiation.  Somewhere the paper
    should discuss this, and whether our inference algorithm can handle it
    (maybe only when written by the programmer?)
    and which types of polymorphism our algorithm can handle and the
    challenges thereto.  This relevant to both \textsc{INVOKE} and \textsc{RETURN}.}

  \todo{If you want to save space, you could drop \textsc{NEW} and say it
    is analogous to \textsc{INVOKE}.  Leaving it in makes the figure
    bigger, which could be a plus or a minus depending on your perspective.}

  \todo{Somewhere, we should show some type-system-specific rules like
    those for \<x.f> and \<a[i]>.}

  \todo{There is a typographic convention that you can use to eliminate all
    the $\forall$ from this diagram.  
$\overline{f_i : q_{F_i}~\tau_{F_i}}$ is shorthand for
$\forall i . ~f_i :~ q_{F_i}~\tau_{F_i}$.  I think this
    would make the diagram easier to read.}

  \caption{Modified type rules used by iterated local type inference.
    Applying these type rules once to every statement in a program
    constitutes the ``inner loop'' of the inference algorithm.
    \std{Gray} indicates
    standard type rules for an object-oriented language with Java-like syntax. \qual{Black} indicates additions to support
    pluggable typechecking\todo{cite (Foster, I guess)}. \infr{Red} indicates additions to support inference, \ie our
    contribution in this paper.
    Throughout, ``R'' subscripts refer to return types; ``F'' to formal parameters; ``A'' to
    actual arguments; and ``I'' to inference results.
    \todo{Which of the qualifier variables can be ``unqualified''?  I guess
      it is all of them?}
    In an assignment \<x=y>, \<x> is the ``formal'' and \<y> is the ``actual''.
    In the \textsc{OVERRIDE} rule, the subscripts ``B'' and ``P'' are mneumonics
    for ``suBtype'' and ``suPertype'', referring the overriding method and the overidden
    method, respectively.
    Type rules that do not require modification to support inference
    are elided for space.}
  \label{fig:type-rules}
\end{figure*}

The modified type rules appear in \cref{fig:type-rules}.
%% This is in the caption, so Mike doesn't think it needs to be in the body text.
% \std{Gray} indicates
% standard type rules for an object-oriented language with Java-like
% language.  \qual{Black} indicates additions to support
% pluggable typechecking. \infr{Red} indicates additions to support inference, \ie our
% contribution in this paper.

$m(f_0,~\ldots,~f_n)$ refers to a method declaration: $m$ is a method name,
and each $f_i$ is a formal parameter declaration.
\todo{Here and in \cref{fig:type-rules}, $f_i$ is ambiguous.  It appears in
  $m(\ldots, f_i, \ldots)$, but it is also mentioned as $f_i~:~q_F~\tau_F$.
  So, does $f_i$ stand for just the name (implied by the latter), or for
  the name and the qualified type (implied by the former)?  Please clarify,
  and use different variables for the two concepts.}
The syntax $m(f_0,~\ldots,~f_n)~:~q_R~\tau_R$ means that $m$'s (qualified) return
type is $q_R~\tau_R$.  The syntax $f_i~:~q_F~\tau_F$ means that the declared
type of the formal parameter $f_i$ (of $m$) is $q_F~\tau_F$.
%
Type rules that do not need to be modified to support inference are elided for space.

$\infEnv$ is the \emph{inference environment}, similar to the standard (qualified)
typing environment $\Gamma$. $\Gamma$ maps expressions and declarations to qualified types.
$\infEnv$ maps declarations to possibly-qualified types.
$\infEnv$ only maps declarations because our inference procedure does not need to infer
types for expressions: we assume that $\Gamma$ already does so via flow-sensitive
refinement. Rather, $\infEnv$'s purpose is to map declarations to the results of inference.
Unlike $\Gamma$, the values in $\infEnv$ are \emph{possibly-qualified}, meaning that they can either
be qualified types or unqualified types. Initially, $\infEnv$ contains qualified types only
for declarations that were qualified before the current type-checking pass of inference (which may come from
the programmer or from a previous type-checking round in the \cref{alg:wpi-fixpoint} fixpoint loop).
Once every statement in the program has been type-checked, the current
round of inference terminates.  Its result is all mappings to qualified
types in $\infEnv$.  Any type that remains unqualified
throughout inference is not annotated, because no information about it was learned.

The function $\mathit{LUB_Q}(q_1, q_2)$ accounts for possibly-qualified types.
$q_1$ and $q_2$ are each either a type qualifier or ``not present''.
If both arguments are qualifiers, then the result of $\mathit{LUB_Q}$
is their least upper bound. If only one qualifier is present, then $\mathit{LUB_Q}$'s result
is that qualifier; if both qualifiers are not present, $\mathit{LUB_Q}$'s result is ``not present'',
resulting in an unqualified result.

\todo{Then, this section should describe the interesting parts of the
  modified type rules in detail? At a minimum, we should probably walk the reader through
  one of the type rules.}

\todo{Within a type-checking round (the ``inner loop''), estimates always
  go up.}


\subsubsection{Completeness}
\label{sec:complete}

Note that this inference system is intentionally \emph{not} complete: it
does not and cannot infer all possibly-true type qualifiers for a given type
system. To see why not, note that type qualifiers on e.g., formal parameters
are inferred from the actual types of the arguments at call sites. If there
are no call sites for a method in a given program, then the \textsc{INVOKE}
rule will never be fired: no information will be inferred for those formal
parameters\todo{not by \textsc{INVOKE}, but \textsc{FORMAL-ASSIGN} might},
and they will remain unqualified.

The lack of completeness is by design. Recall that our goal is not a
set of type qualifiers that perfectly captures whatever facts are true
about the program, but rather a set of type qualifiers that is useful
\emph{in practice} for typechecking programs. Not all true type qualifiers
are useful; as our experiments show, this system still produces many more
(true) type qualifiers than a human would write. \todo{Add a forward ref
  to the prior sentence.} Another benefit of completeness being a non-goal
is that our inference system is permitted to not infer a type qualifier
in practically any scenario where doing so might lead to sub-optimal results;
for example, see our handling of recursion (\cref{sec:infinite-descending-chains}).

% LocalWords:  typechecker typechecked typechecking typecheckers intra
% LocalWords:  decl standard''
