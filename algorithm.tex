\section{Type Inference Algorithm}
\label{sec:algorithm}

This section presents our type inference algorithm. This algorithm is
polymorphic with respect to the underlying pluggable typechecker: that is,
it applies equally well to any pluggable typechecker that performs flow-sensitive
local inference. The algorithm is described in two parts: first, the fixpoint
algorithm used to infer types for a particular program (\cref{sec:core-algorithm});
and second, the modifications we make to the pluggable typechecker to enable
that algorithm (\ie the \textsc{Instrument} function of \cref{alg:wpi-fixpoint},
in \cref{sec:instrument}).

\subsection{Fixpoint Algorithm}
\label{sec:core-algorithm}

% the core WPI fixpoint algorithm; that is, the outer WPI loop

\begin{algorithm}
  \DontPrintSemicolon
  \SetKwFunction{Infer}{infer}
  \SetKwProg{Fn}{def}{:}{}
  \SetKwInOut{Input}{input}
  \SetKwInOut{Output}{output}
  \Input{program $P$ and pluggable typechecker $T$}
  \Output{set of errors $E$ and set of type qualifiers $A$}
  \Fn{ \Infer{$P$, $T$} } {
%    \tcc{In each iteration, $\|prevA|$ is the specification set before running $T$, and $A$ is the set after.}
%    $\|prevA| \gets \emptyset, A \gets \emptyset$ \;
    $A \gets \emptyset$ \;
%    \tcc{\textsc{Instrument} modifies $T$ to both check the program and collect a candidate specification set.}
    $T_I \gets \textsc{EnableInference}(T)$ \;
%    $E, A \gets T_I(P, \|prevA|)$ \;
    \Repeat{$E = \emptyset \vee \|prevA| = A$}{
      % \tcc{Note that each element of $A$ is either a new type qualifier or refines some element of $\|prevA|$.}
      $\|prevA| \gets A$ \;
      $E, A \gets T_I(P, \|prevA|)$ \;
    }
    \Return $E, A$ \;
  }
  \caption{Iterated local type inference algorithm.  This is the ``outer
    loop'' of the approach, which iterates to a fixed point.
    The helper function \textsc{EnableInference} is defined by the modifications
    to the framework described in \cref{sec:instrument}.
    \todo{Why is the caption so narrow, not taking up the whole column?}
}
  \label{alg:wpi-fixpoint}
\end{algorithm}


This section presents the core fixpoint algorithm, which appears
in \cref{alg:wpi-fixpoint}. The key idea is to iteratively analyze
the target program ($P$) with an instrumented version of the
pluggable typechecker, recording intermediate results at each
step (the sets of specifications $A$ and $A^{\prime}$) until
either the program can be typechecked (\ie $E \neq \emptyset$,
meaning there are no remaining typechecking errors) or the
specifications reach fixpoint (\ie $A = A^{\prime}$) and
no further refinement is possible.

This fixpoint loop is quite general: its success for our
purposes depends heavily on the \textsc{Instrument} procedure
that outputs locally-inferred specifications. \Cref{sec:instrument}
describes how we implement \textsc{Instrument} in a way that is
applicable to any pluggable typechecker.

\subsubsection{Soundness}
\label{sec:soundness}

Nevertheless, \cref{alg:wpi-fixpoint} is sound, assuming the underlying typechecker $T$
is sound. The key intuition is that \cref{alg:wpi-fixpoint} always runs $T$
on each set of proposed specifications. If any specification is incorrect,
$T$ will report an error---in the same manner as if a human had written an
incorrect specification. In particular, this means that the instrumentation
(\cref{sec:instrument}) has no obligation to produce annotations that are sound,
so long as the underlying typechecking algorithm itself is not modified.
\todo{Maybe formalize this a bit more?}

\subsection{Modifications to the Typechecker}
\label{sec:instrument}

The algorithm presented in \cref{sec:core-algorithm} works for
any pluggable typechecker that supports flow-sensitive inference:
that is, it does not require a type system implementer to write
any special rules to support type inference. Instead, this section
describes our approach to \emph{automatically} modify a given
pluggable typechecker to support inference, corresponding to the
\textsc{Instrument} helper function in \cref{alg:wpi-fixpoint}.

The key idea behind our approach to instrumenting the typechecker
is to do the instrumentation at the \emph{framework} level: that is
we modify the underlying infrastructure that typecheckers rely on
once, and the modifications can then be applied generically to any
typechecker compatible with the original framework. Our modifications
can be conceptualized at the type-qualifier-theory level: that is,
we modify the rules for typechecking \emph{any} pluggable type system
so that inference is supported,
regardless of the particular qualifiers it happens to support.

\todo{Create a figure with modified type rules for handling
  inference, with one rule for each part of \<WholeProgramInference>.
  We want to present this as a theoretical contribution: \ie, as
  a new way of building a pluggable type system. Maybe take the original
  rules from~\cite{FosterFFA99}?}

\todo{Then, this section should describe the interesting parts of the
  modified type rules in detail.}
