%%
%% This is file `sample-sigplan.tex',
%% generated with the docstrip utility.
%%
%% The original source files were:
%%
%% samples.dtx  (with options: `sigplan')
%% 
%% IMPORTANT NOTICE:
%% 
%% For the copyright see the source file.
%% 
%% Any modified versions of this file must be renamed
%% with new filenames distinct from sample-sigplan.tex.
%% 
%% For distribution of the original source see the terms
%% for copying and modification in the file samples.dtx.
%% 
%% This generated file may be distributed as long as the
%% original source files, as listed above, are part of the
%% same distribution. (The sources need not necessarily be
%% in the same archive or directory.)
%%
%% The first command in your LaTeX source must be the \documentclass command.
\documentclass[sigconf,screen,review,anon]{acmart}
\settopmatter{printccs=true, authorsperrow=4}

%%
%% \BibTeX command to typeset BibTeX logo in the docs
\AtBeginDocument{%
  \providecommand\BibTeX{{%
    \normalfont B\kern-0.5em{\scshape i\kern-0.25em b}\kern-0.8em\TeX}}}

%% Rights management information.  This information is sent to you
%% when you complete the rights form.  These commands have SAMPLE
%% values in them; it is your responsibility as an author to replace
%% the commands and values with those provided to you when you
%% complete the rights form.
% \setcopyright{acmcopyright}
% \copyrightyear{2018}
% \acmYear{2018}
% \acmDOI{10.1145/1122445.1122456}

%% These commands are for a PROCEEDINGS abstract or paper.
% \acmConference[Woodstock '18]{Woodstock '18: ACM Symposium on Neural
%   Gaze Detection}{June 03--05, 2018}{Woodstock, NY}
% \acmBooktitle{Woodstock '18: ACM Symposium on Neural Gaze Detection,
%   June 03--05, 2018, Woodstock, NY}
% \acmPrice{15.00}
% \acmISBN{978-1-4503-9999-9/18/06}


%%
%% Submission ID.
%% Use this when submitting an article to a sponsored event. You'll
%% receive a unique submission ID from the organizers
%% of the event, and this ID should be used as the parameter to this command.
%%\acmSubmissionID{123-A56-BU3}

%%
%% The majority of ACM publications use numbered citations and
%% references.  The command \citestyle{authoryear} switches to the
%% "author year" style.
%%
%% If you are preparing content for an event
%% sponsored by ACM SIGGRAPH, you must use the "author year" style of
%% citations and references.
%% Uncommenting
%% the next command will enable that style.
%%\citestyle{acmauthoryear}


% tables
\usepackage[utf8]{inputenc} 
\usepackage[T1]{fontenc}
\usepackage{microtype}
\usepackage{tabularx}
\usepackage{multirow}
\usepackage{booktabs}

\usepackage[ruled,vlined]{algorithm2e}

% margin notes
\usepackage{marginnote}

% xspace command
\usepackage{xspace}

% Reduce font size in verbatim environment
\usepackage{fancyvrb}
\RecustomVerbatimEnvironment{Verbatim}{Verbatim}{fontsize=\smaller}

% lstlisting command
\usepackage{listings}
\usepackage[scaled]{beramono}
\newcommand*\LSTfont{\Small\fontencoding{T1}\ttfamily\SetTracking{encoding=*}{-60}\lsstyle}
\lstset{language=Java,
  frame=none,
  aboveskip=1.5pt,
  belowskip=0pt,
  showstringspaces=false,
  columns=flexible,
  basicstyle=\LSTfont,
  numbers=none,
  numberstyle=\tiny\color{black},
  keywordstyle=\color{black},
  commentstyle=\color{black},
  stringstyle=\color{black},
  breaklines=true,
  breakatwhitespace=true,
  tabsize=3,
  %emph={@NonNegative,@Positive,@GTENegativeOne,@LTLengthOf,@LTEqLengthOf,@IndexFor,@IndexOrHigh,@IndexOrLow,@HasSubsequence,@LessThan,@SameLen,@SearchIndexFor,@MinLen,@ArrayLen,@IntVal,@IntRange,@LengthOf,@UpperBoundUnknown,@LowerBoundUnknown,int,double,List,Map,Object,SerialDate,Long,Integer,DefaultPolarItemRenderer,LegendItem,PolarPlot,XYDataset,long,T,String,string,byte,InputStream,CategoryDataset,DatasetRenderingOrder,ArrayList,Entry,Values,Number,ValuesContract,ImmutableIntArray,Dataset,XYZDataset}, emphstyle=\color{blue}
}

% Graphics
\usepackage{tikz}
\usetikzlibrary{arrows,automata,positioning}

% inference rules
\usepackage{mathpartir}

% Change font and line spacing for figure captions
% \usepackage{setspace,caption}
% \captionsetup{labelfont={small,bf}, textfont={small,bf,stretch=0.8}, labelsep=colon, margin=0pt}

\usepackage{balance} % balanced columns on last page

% cref command; best to load last
\usepackage{cleveref}
\newcommand{\crefrangeconjunction}{--}

%%% The following is specific to ESEC/FSE '21 and the paper
%%% 'Lightweight and Modular Resource Leak Verification'
%%% by Martin Kellogg, Narges Shadab, Manu Sridharan, and Michael D. Ernst.
%%%

\setcopyright{rightsretained}
\acmPrice{}
\acmDOI{10.1145/3468264.3468576}
\acmYear{2023}
\copyrightyear{2023}
\acmSubmissionID{foo}
\acmISBN{foo}
\acmConference[Under review '22]{Under review}{Date}{Location}
\acmBooktitle{Under review}

%%
%% end of the preamble, start of the body of the document source.
\begin{document}

%%% Todo comments
\newcommand{\todo}[1]{{\color{red}\bfseries [[#1]]}}
%% Comment or uncomment this line.
% \renewcommand{\todo}[1]{\relax}

\newcommand{\manu}[1]{\todo{#1 --MS}}

% Don't show todo commands if this macro is defined.
\ifdefined\notodocomments
  \renewcommand{\todo}[1]{\relax}
\fi

\newif\ifanonymous
% \anonymoustrue
\anonymousfalse

\newcommand{\anonurl}[1]{\ifanonymous URL removed for anonymity.\else\url{#1}\fi}
\newcommand{\footnoteanonurl}[1]{\footnote{\anonurl{#1}}}

%% The name of our tool.
\newcommand{\Tool}{The Resource Leak Checker\xspace}
\newcommand{\tool}{the Resource Leak Checker\xspace}

% \|name| or \mathid{name} denotes identifiers and slots in formulas
\def\|#1|{\mathid{#1}}
\newcommand{\mathid}[1]{\ensuremath{\mathit{#1}}}
% \<name> or \codeid{name} denotes computer code identifiers
\def\<#1>{\codeid{#1}}
% \protected\def\codeid#1{\ifmmode{\mbox{\sf{#1}}}\else{\sf #1}\fi}
% \protected\def\codeid#1{\ifmmode{\mbox{\ttfamily{#1}}}\else{\ttfamily #1}\fi}
\protected\def\codeid#1{\ifmmode{\mbox{\smaller\ttfamily{#1}}}\else{\smaller\ttfamily #1}\fi}

\newcommand{\CalledMethodsBottom}{\<@Call\-ed\-Meth\-ods\-Bottom>\xspace}
\newcommand{\CalledMethods}{\<@Call\-ed\-Meth\-ods>\xspace}
\newcommand{\EnsuresCalledMethods}{\<@En\-sures\-Call\-ed\-Meth\-ods>\xspace}
\newcommand{\MustCall}{\codeid{@Must\-Call}\xspace}
\newcommand{\MustCallAlias}{\codeid{@Must\-Call\-Alias}\xspace}
\newcommand{\MustCallUnknown}{\codeid{@Must\-Call\-Unknown}\xspace}
\newcommand{\CreatesMustCallFor}{\<@Creates\-Must\-Call\-For>\xspace}
% Deprecated
\newcommand{\ResetMustCall}{\CreatesMustCallFor}

% "trule" stands for ``type rule''
\newcommand{\trule}[2]{\[\frac{#1}{#2}\]}
\newcommand{\truleinline}[2]{\ensuremath{#1\mathrel{\vdash}#2}}
\newcommand{\hastype}[1]{\mathbin{:}\trtext{#1}}
\newcommand{\trcode}[1]{\codeid{\smaller\smaller #1}}
\newcommand{\trtext}[1]{\mbox{\smaller\smaller #1}}
\newcommand{\trquoted}[1]{\trcode{"}#1\trcode{"}}


\hyphenation{type-state}        % LaTeX defaults to "types-tate"

%%% Space-saving hacks

% Reduce indentation in lists.
\setlength{\leftmargini}{.75\leftmargini}
\setlength{\leftmarginii}{.75\leftmarginii}
\setlength{\leftmarginiii}{.75\leftmarginiii}

\newcommand{\prefigcaption}{\vspace{-1pt}}
\newcommand{\posttablecaption}{\vspace{-1pt}}

% Reduce the separation between figures and text.
\addtolength{\textfloatsep}{-.25\textfloatsep}
\addtolength{\dbltextfloatsep}{-.25\dbltextfloatsep}
\addtolength{\floatsep}{-.25\floatsep}
\addtolength{\dblfloatsep}{-.25\dblfloatsep}

\newcommand{\zph}{\phantom{0}}
\newcommand{\zzph}{\phantom{0}}

\newcommand{\ie}{\textit{i.e.},\xspace}
\newcommand{\eg}{\textit{e.g.},\xspace}
\newcommand{\etc}{\textit{etc.}\xspace}
\newcommand{\etal}{\textit{et al.}\xspace}
\newcommand{\aka}{\textit{a.k.a.}\xspace}	


% LocalWords:  acks


%%
%% The "title" command has an optional parameter,
%% allowing the author to define a "short title" to be used in page headers.
\title{Type Inference for Pluggable Typecheckers}
% or: \title{Inference of Pluggable Types}

%%
%% The "author" command and its associated commands are used to define
%% the authors and their affiliations.
%% Of note is the shared affiliation of the first two authors, and the
%% "authornote" and "authornotemark" commands
%% used to denote shared contribution to the research.

\author{Martin Kellogg}
\affiliation{New Jersey Institute of Technology, Newark
 \country{USA}}
\email{martin.kellogg@njit.edu}

\author{Michael D. Ernst}
\affiliation{University of Washington, Seattle
  \country{USA}}
\email{mernst@cs.washington.edu}

%%
%% By default, the full list of authors will be used in the page
%% headers. Often, this list is too long, and will overlap
%% other information printed in the page headers. This command allows
%% the author to define a more concise list
%% of authors' names for this purpose.
%\renewcommand{\shortauthors}{Trovato and Tobin, et al.}

%%
%% The abstract is a short summary of the work to be presented in the
%% article.
\begin{abstract}
  A pluggable type system extends a host programming language with type qualifiers.
  It enables a programmer to write a type such as \<unsigned int>, \<secret
  string>, or \<nonnull object>.
  Type-checking with pluggable types detects and prevents more errors than
  the host type system.
  Type inference can solve the biggest obstacle to use of pluggable types: the need for
  programmers to write type qualifiers.
  Traditional approaches to type inference are type-system-specific:
  build and then solve a system
  of constraints corresponding to the rules of the underlying type system.
  For each new pluggable type system, the type inference algorithm
  must be extended to build and then solve constraints specific
  to that pluggable type system.

  We propose a novel type inference algorithm
  that can infer type qualifiers for \emph{any} pluggable
  type system with no new\todo{This is a bit strong.  We do need custom
    code to deal with special cases (the framework has hooks for some of
    these), including non-qualifiers (declaration
    annotations, pre- and post-conditions.} type-system-specific code. The key insight behind
  our new technique is that extant, practical pluggable type systems are
  \emph{flow-sensitive}
  and therefore include \emph{local} type inference components. Our algorithm
  uses the results of local type inference to produce candidate summaries
  for procedures. These new summaries enable new local inference results
  elsewhere in the program---our algorithm runs until a fix-point in procedure
  summaries is reached.

  We have implemented our algorithm for the open-source Checker Framework
  project, which is widely used in industry and on which dozens of specialized pluggable typecheckers
  have been built. In experiments with \numTypeSystems distinct
  pluggable type systems and \numProjects projects, our algorithm
  inferred \percentInferred of the annotations that programmers wrote. We also compared our
  technique to other techniques for type inference.
\end{abstract}

%%
%% The code below is generated by the tool at http://dl.acm.org/ccs.cfm.
%% Please copy and paste the code instead of the example below.
%%
\begin{CCSXML}
<ccs2012>
<concept>
<concept_id>10011007.10011074.10011099</concept_id>
<concept_desc>Software and its engineering~Software verification and validation</concept_desc>
<concept_significance>500</concept_significance>
</concept>
</ccs2012>
\end{CCSXML}

%\ccsdesc[500]{Software and its engineering~Software verification and validation}

\ccsdesc[500]{Software and its engineering~Software verification}

%%
%% Keywords. The author(s) should pick words that accurately describe
%% the work being presented. Separate the keywords with commas.
\keywords{Pluggable type systems, type qualifiers, type checking, type inference, static analysis}

%% A "teaser" image appears between the author and affiliation
%% information and the body of the document, and typically spans the
%% page.
%% \begin{teaserfigure}
%%   \includegraphics[width=\textwidth]{sampleteaser}
%%   \caption{Seattle Mariners at Spring Training, 2010.}
%%   \Description{Enjoying the baseball game from the third-base
%%   seats. Ichiro Suzuki preparing to bat.}
%%   \label{fig:teaser}
%% \end{teaserfigure}

%%
%% This command processes the author and affiliation and title
%% information and builds the first part of the formatted document.

\maketitle

\section{Introduction}
\label{sec:intro}

\todo{Explain the distinction between a ``specification'' and a type qualifier.}

\todo{Should we be more careful to distinguish type annotations (a
  Java syntactic construct) from type qualifiers (a type-theoretic
  concept)?}

A pluggable type system~\cite{FosterFFA99} augments a host type system
with \emph{type qualifiers} that refine it.  A qualified type is
finer-grained than an unqualified one and therefore gives more precise
information about what values are possible at run time.
Researchers have devised practical pluggable type systems
to prevent numerous kinds of defects, including null-pointer
dereferences~\cite{BanerjeeCS2019,PapiACPE2008,DietlDEMS2011},
array-bounds violations~\cite{KelloggDME2018},
violations of locking discipline~\cite{ErnstLMST2016},
mutations of immutable data~\cite{DietlDEMS2011,PapiACPE2008,coblenz2017glacier},
and others.
A successful type-checking run proves that the undesirable behavior will
never occur.
Pluggable type systems are a standard practice in industry; for example,
they are used at Amazon~\cite{KelloggSTE2020,KelloggRSSE2020},
Google~\cite{SadowskiAEMCJ2018}, and Uber~\cite{BanerjeeCS2019}.

Pluggable types are an attractive verification and bug-finding strategy
because programmers
are familiar with type systems and are used to writing types.
% because
% they are programming in a host language like Java or C\# with static
% types.
The type qualifiers also serve as concise, machine-checked documentation.
However, writing type qualifiers in a legacy codebase
is time-consuming and intimidating for developers.

The traditional approach to solving this problem
is \emph{type inference}:
deducing the proper types for a program from the program's structure
by solving a set of constraints induced by type uses.
%
For example, languages with Hindley-Milner type systems like ML and Haskell
include type inference (based on algorithm W~\cite{DamasM1982})
in their compilers, so programmers do not need to write types at all.
%
This approach is impractical for popular object-oriented programming
languages such as Java or Python, because subtyping prevents efficient
unification of types (\todo{say something about row polymorphism as a
  possible alternative, as is done in OCaml, here?}).
%
Type inference for dynamically-typed languages like Python is an open
research problem. Researchers have proposed approaches based
on MaxSAT~\cite{hassan2018maxsmt} and machine learning~\cite{xu2016python,peng2022static}.

Unfortunately, none of these approaches is obviously practical for pluggable types.
%% Actually, this is basically what we're going to do, so I shouldn't be so harsh on it.
%% Building a type inference algorithm in the style of algorithm W for each
%% type system is impractical: there are too many pluggable type
%% systems.\todo{I don't buy this.  There aren't too many type systems to
%%   implement.  Doing a bit of extra work to define inference for each is a
%%   constant factor overhead over writing the type checker for each.}
A natural idea is to build a set of constraints based on the property the
pluggable type system is trying to prove (\eg, build constraints based on
nullability for a nullness pluggable type system, constraints based on indexing
for an array bounds system, \etc) and then dispatch those constraints
to a solver.
%
Unfortunately, this approach requires significant work for every pluggable type
system (as all of its type rules must be re-encoded as constraints) and rapidly
reduces to building an alternative verification tool, making it unattractive as
a general solution to the inference problem.
%
Approaches based on machine learning lack sufficient training data for
many pluggable type systems, they give no guarantee of correctness, and may
not be able to explain their choices to the programmer.
%
We desire an
inference algorithm that is \emph{generic} over the pluggable type system
to which it is applied in that same way that algorithm W is generic over
all Hindley-Milner type systems: the pluggable type system designer should not need to
modify their pluggable type system's implmentation in order to access
the benefits of type inference.

To that end, we propose a general type inference algorithm for pluggable type
systems that is applicable to any flow-sensitive pluggable type system.
Our key insight is that
practical frameworks for building pluggable type systems already provide
local type inference in the form of flow-sensitivity within the body
of methods.
We modify the framework to record inferred method, class, and field summaries
based on the results of flow-sensitive typechecking.
Our approach iteratively typechecks the program, using and improving the
summaries, until reaching a fixed point.
That fixed point is a candidate set of type qualifiers, which are
consistent with the program.

If the program typechecks with the candidate set of qualifiers, the program
is correct with respect to that type system.
If not, then the program contains a defect, or it is not internally
consistent, or its correctness is beyond the capabilities of the
typechecker (this is when a programmer would write a cast).
In any event, the inferred type qualifiers can help the programmer.
In our implementation, the user (a programmer) can decide whether to insert
the type qualifiers in the source code, or to store them in a side file.

We implemented this idea for the Checker Framework~\cite{PapiACPE2008},
a popular open-source pluggable type system
framework for Java.
%
\todo{There were some difficulties along the way---we're going to
impress you with our clever solution to tricky problems, and also how much
engineering we did.}
%
We used the extended Checker Framework to run \todo{X} different pluggable typecheckers
on \todo{some massive number of} lines of code to demonstrate that
our approach is both effective and general.
In experiments, our inference approach inferred \todo{Y\%} of human-written
type qualifiers.

\todo{Decide if we're actually going to do this.}
We also compared our approach to bespoke inference systems that build
systems of constraints for particular type systems, like nullness or
maybe CFI (if Werner is an author of this paper). We show that our approach
produces similarly-good results, but didn't require a ton of extra
implementation effort for each new type system.

Our contributions are:
\begin{itemize}
\item a novel type inference algorithm for flow-sensitive pluggable
  typecheckers (\cref{sec:core-algorithm});
\item a collection of enhancements to the algorithm that are necessary to
  make it practical (\cref{sec:difficulties});
\item an implementation of our new type inference algorithm within a framework
  for building pluggable typecheckers (\cref{sec:implementation});
\item an evaluation of our implementation, that shows that it can infer
  \todo{X}\% of human-written annotations in \todo{Y} projects totalling
  \todo{Z} lines of non-comment, non-blank Java code, across \todo{W} different
  pluggable typecheckers (\cref{sec:evaluation}); and,
\item a comparison of our generic algorithm to specialized inference
  techniques for specific typecheckers, which demonstrates that our generic
  approach is about as good at inferring annotations but requires less
  custom code (\cref{sec:comparison}).
\end{itemize}


\section{Background and Motivating Example}
\label{sec:motivating-example}

Types are sets of run-time values that
overestimate possible run-time values. 
A \textit{type qualifier}~\cite{FosterFFA99}
is a restriction on a type that limits which run-time value
the qualified type can represent. For example, \<positive int>
is a qualifier type: \<positive> is the type qualifier, and \<int>
is the base type.
%
A pluggable type system consists of a hierarchy of type qualifiers.

In practice, pluggable type systems are flow-sensitive.
For example, after a test \codeid{x.f > 0}, the type of \<x.f>
might change from \<int> to \<positive int>
until a possible side effect or a control flow join.
%
Each pluggable typechecker is therefore effectively
an abstract interpretation~\cite{Cousot1997}, with the
abstract interpretation's lattice being equivalent to
the type qualifier hierarchy.

Pluggable type systems in practice are also modular.
For example, consider the following example:

\begin{verbatim}
/** Returns the value in a at the index. The first
    element of the array is index 1. */
int getOneIndexed(int[] a, int index) {
  return a[i - 1];
}
\end{verbatim}

A pluggable typechecker designed to prevent negative array accesses
might require that the type of any array access is non-negative
(assuming that arrays are actually zero-indexed).
%
Because \<index>'s type is an unqualified type from the host language
(\ie a ``base type''),
a pluggable typechecker would \emph{default}
it to a qualifier modeling the worst-case assumption: that \<index> could
be any integer (in type system terminology, this would be referred to
as the ``top qualifier'' or $\top$).
%
To typecheck this code,
a programmer would therefore need to change the type of \<index> from
\<int> to \<positive int> by adding the type qualifier \<positive>:
the typechecker, being modular, would not automatically reason across procedure boundaries.

Our goal in this work is to avoid the burden of writing these type qualifiers
by automatically taking a flow-sensitive, modular pluggable typechecker
like the ones that exist in practice and transforming it into one that performs
inter-procedural inference. Though in this example it was easy to write
the single necessary \<positive> type qualifier, when using practical pluggable
type systems the annotation burden scales linearly with the size of the code
base, and many pluggable type systems require significant numbers of annotations.
For example, a pluggable type system for preventing out-of-bounds array accesses
(with an equivalent to the \<positive> type qualifier from this example) required
one type qualifier for every 32 lines of non-comment, non-blank code~\cite{KelloggDME2018}.


\section{Type Inference Algorithm}
\label{sec:algorithm}

This section presents our type inference algorithm. This algorithm is
independent of the underlying pluggable typechecker: that is,
it applies equally well to any pluggable typechecker that performs flow-sensitive
local inference.

\todo{I don't like the term ``instrumented''.  It does describe what
  happened at a low level.  It does not explain what the instrumentation
  does or why it is valuable.  Maybe an ``inferring'' version of the type checker?}
\todo{The paper should explain that ``specifications'' is really ``type
  annotations'', to a first approximation.}

The key idea is to modify the underlying \emph{framework}
on which the target pluggable typecheckers are built. Pluggable
typecheckers use a modified version of the host type system
that supports type qualifiers. Our approach modifies this support
layer for pluggable typechecking in order to support inference for
any typechecker.

A typechecker $T : P \rightarrow E$
takes a program and outputs a (possibly empty) set of
type errors.  A typechecker running using our modified framework
$T' : \langle P, A \rangle \rightarrow \langle E, A' \rangle$
takes a program along with a set of type qualifiers $A$, and outputs errors and
inferences (\ie $A'$, a new set of type qualifiers).
The errors $E$ are exactly those $T$ would output, if the
type qualifiers in $A$ had been written on $P$ by a programmer.

We describe the algorithm in two parts: first,
\Cref{sec:core-algorithm} gives the fixpoint
algorithm used to infer types for a particular program
using our modified typechecking framework.
\Cref{sec:instrument} explains the modifications to the pluggable
typechecking framework that enable type inference for a pluggable
typechecker $T$ (\ie convert it to $T'$ in \cref{alg:wpi-fixpoint}).


\subsection{Fixpoint Algorithm}
\label{sec:core-algorithm}

% the core WPI fixpoint algorithm; that is, the outer WPI loop

\begin{algorithm}
  \DontPrintSemicolon
  \SetKwFunction{Infer}{infer}
  \SetKwProg{Fn}{def}{:}{}
  \SetKwInOut{Input}{input}
  \SetKwInOut{Output}{output}
  \Input{program $P$ and pluggable typechecker $T$}
  \Output{set of errors $E$ and set of type qualifiers $A$}
  \Fn{ \Infer{$P$, $T$} } {
%    \tcc{In each iteration, $\|prevA|$ is the specification set before running $T$, and $A$ is the set after.}
%    $\|prevA| \gets \emptyset, A \gets \emptyset$ \;
    $A \gets \emptyset$ \;
%    \tcc{\textsc{Instrument} modifies $T$ to both check the program and collect a candidate specification set.}
    $T_I \gets \textsc{EnableInference}(T)$ \;
%    $E, A \gets T_I(P, \|prevA|)$ \;
    \Repeat{$E = \emptyset \vee \|prevA| = A$}{
      % \tcc{Note that each element of $A$ is either a new type qualifier or refines some element of $\|prevA|$.}
      $\|prevA| \gets A$ \;
      $E, A \gets T_I(P, \|prevA|)$ \;
    }
    \Return $E, A$ \;
  }
  \caption{Iterated local type inference algorithm.  This is the ``outer
    loop'' of the approach, which iterates to a fixed point.
    The helper function \textsc{EnableInference} is defined by the modifications
    to the framework described in \cref{sec:instrument}.
    \todo{Why is the caption so narrow, not taking up the whole column?}
}
  \label{alg:wpi-fixpoint}
\end{algorithm}


This section presents the core fixpoint algorithm, which appears
in \cref{alg:wpi-fixpoint}. The key idea is to iteratively analyze
the target program ($P$) with a version of the
pluggable typechecker whose core qualified type rules
have been modified to support inference in the manner described in
\cref{sec:instrument}, recording intermediate results at each
step (the sets of inferred type qualifiers $A$ and $A^{\prime}$) until
either there are no remaining typechecking errors
(\ie $E \neq \emptyset$)
or the
type qualifiers reach fixpoint (\ie $A = A^{\prime}$).

%% This fixpoint loop is quite general: its success for our
%% purposes depends heavily on the \textsc{Instrument} procedure
%% that outputs locally-inferred specifications. \Cref{sec:instrument}
%% describes how we implement \textsc{Instrument} in a way that is
%% applicable to any pluggable typechecker.

Note that this algorithm monotonically refines types---\ie
each element $a^{\prime}$ of $A^{\prime}$ is either 1) not present in $A$, or
2) there exists some element $a$ in $A$ such that $a^{\prime} \sqsubseteq a$
and both $a$ and $a^{\prime}$ qualify the same base type in the same program location.
The proof of this is by induction on the type rules in \cref{sec:instrument}.

\subsubsection{Soundness}
\label{sec:soundness}

Any inference algorithm is sound so long as the typechecker is run afterward:
even if the inference algorithm were to produce incorrect type qualifiers,
the typechecker would reject them.

\subsection{Modifications to the Typechecker}
\label{sec:instrument}

The algorithm presented in \cref{sec:core-algorithm} works for
any pluggable typechecker that supports flow-sensitive inference:
that is, it does not require a type system implementer to write
any special rules to support type inference. Instead, this section
describes our approach to \emph{automatically} modify a given
pluggable typechecker to support inference, corresponding to the
\textsc{Instrument} helper function in \cref{alg:wpi-fixpoint}.

The key idea behind our inference approach to instrumenting the typechecker
is to do modify the \emph{framework}.  Once the pluggable type-checking
framework is modified, inference is enabled for every typechecker built on it.
Our modifications
can be conceptualized at the type-qualifier-theory level: that is,
we modify the rules for typechecking \emph{any} pluggable type system
so that inference is supported,
regardless of the particular qualifiers it happens to support.

Our modifications rely on the fact that practical pluggable type systems
do \emph{local}, intra-procedural flow-sensitive type inference.
This means that programmers rarely need to write annotations within method bodies.
Analogously, Java's \<var> keyword permits programmers to omit Java basetypes within
method bodies.
Programmers are more willing to write types on method
signatures, where they form valuable documentation.  (Our goal is to lift
even this burden, for type annotations)

This assumption is
reasonable in practice: programmers are generally not willing to
write type qualifiers within method bodies, and so 
the pluggable type frameworks that exist in practice \todo{all?}
support this feature~\cite{PapiACPE2008}~\todo{cite any other practical pluggable
  type frameworks that exist, if there are any}. \todo{Also mention that
  Java itself supports this now for the base type system?} \todo{Also mention
  that this seems to be a general trend in language design, citing maybe Kotlin?}

\begin{figure*}
  \begin{mathpar}
    \inferrule* [right=INVOKE]
                {
                  \std{\Gamma \vdash m(f_0,\ldots,f_n) : }~\qual{q_R}~\std{\tau_R}
                  \\
                  \std{\Gamma \vdash \forall i \in 0,\ldots,n . ~e_i :~} \qual{q_{i_A}}~\std{\tau_{i_A}}
                  \\
                  \std{\Gamma \vdash \forall i \in 0,\ldots,n . ~f_i :~} \qual{q_{i_F}}~\std{\tau_{i_F}}
                  \\
                  \std{\Gamma \vdash \forall i \in 0,\ldots,n . ~} \qual{q_{i_A}}~\std{\tau_{i_A}~\sqsubseteq}~\qual{q_{i_F}}~\std{\tau_{i_F}}
                  \\
                  \infr{\infEnv \vdash \forall i \in 0,\ldots,n . ~f_i~:~q_{i_I}~\tau_{i_F}}
                }
                {
                  \std{\Gamma \vdash m(e_0,\ldots,e_n) : }~\qual{q_R}~\std{\tau_R}
                  \\
                  \infr{\infEnv \vdash \forall i \in 0,\ldots,n . ~f_i~:~\mathit{LUB_Q}(q_{i_A},~q_{i_I})~\tau_{i_F} }
                }
                
     \inferrule* [right=NEW]
                {
                  \std{\Gamma \vdash \<new T>(f_1,\ldots,f_n) : }~\qual{q_R}~\std{\tau_R}
                  \\
                  \std{\Gamma \vdash \forall i \in 1,\ldots,n . ~e_i :~} \qual{q_{i_A}}~\std{\tau_{i_A}}
                  \\
                  \std{\Gamma \vdash \forall i \in 1,\ldots,n . ~f_i :~} \qual{q_{i_F}}~\std{\tau_{i_F}}
                  \\
                  \std{\Gamma \vdash \forall i \in 1,\ldots,n . ~} \qual{q_{i_A}}~\std{\tau_{i_A}~\sqsubseteq}~\qual{q_{i_F}}~\std{\tau_{i_F}}
                  \\
                  \infr{\infEnv \vdash \forall i \in 1,\ldots,n . ~f_i~:~q_{i_I}~\tau_{i_F}}
                }
                {
                  \std{\Gamma \vdash \<new T>(e_1,\ldots,e_n) : }~\qual{q_R}~\std{\tau_R}
                  \\
                  \infr{\infEnv \vdash \forall i \in 1,\ldots,n . ~f_i~:~\mathit{LUB_Q}(q_{i_A},~q_{i_I})~\tau_{i_F} }
                }

     \inferrule* [right=FORMAL-ASSIGN]
                {
%                  \std{f~is~a~formal~parameter} \\
                  \std{\Gamma \vdash f~:}~\qual{q_F}~\std{\tau_F} \\
                  \std{\Gamma \vdash e~:}~\qual{q_A}~\std{\tau_A} \\
                  \std{\Gamma \vdash} \qual{q_A}~\std{\tau_A~\sqsubseteq}~\qual{q_F}~\std{\tau_F} \\
                  \infr{\infEnv \vdash f~:~q_I~\tau_F}
                }
                {
                  \std{\Gamma \vdash f~:=~e} \\
                  \infr{\infEnv \vdash f~:~\mathit{LUB_Q}(q_A, q_I)~\tau_F}
                }

     \inferrule* [right=FIELD-ASSIGN]
                {
%                  \std{f~is~a~formal~parameter} \\
                  \std{\Gamma \vdash x.f~:}~\qual{q_F}~\std{\tau_F} \\
                  \std{\Gamma \vdash e~:}~\qual{q_A}~\std{\tau_A} \\
                  \std{\Gamma \vdash} \qual{q_A}~\std{\tau_A~\sqsubseteq}~\qual{q_F}~\std{\tau_F} \\
                  \infr{\infEnv \vdash x.f~:~q_I~\tau_F}
                }
                {
                  \std{\Gamma \vdash x.f~:=~e} \\
                  \infr{\infEnv \vdash x.f~:~\mathit{LUB_Q}(q_A, q_I)~\tau_F}
                }

     \inferrule* [right=RETURN]
                {
%                  \std{f~is~a~formal~parameter} \\
                  \std{\Gamma \vdash m(f_0,\ldots,f_n)~:}~\qual{q_F}~\std{\tau_F} \\
                  \std{\Gamma \vdash e~:}~\qual{q_A}~\std{\tau_A} \\
                  \std{\Gamma \vdash} \qual{q_A}~\std{\tau_A~\sqsubseteq}~\qual{q_F}~\std{\tau_F} \\
                  \infr{\infEnv \vdash m(f_0,\ldots,f_n)~:~q_I~\tau_F}
                }
                {
%                  \std{\Gamma \vdash \<return>~e} \\
                  \std{\<return>~e \in m} \\
                  \infr{\infEnv \vdash m(f_0,\ldots,f_n)~:~\mathit{LUB_Q}(q_A, q_I)~\tau_F}
                }

    \inferrule* [right=OVERRIDE]
                {
                  % return types
                  \std{\Gamma \vdash m_1(f_{0_1},\ldots,f_{n_1}) : }~\qual{q_{R_1}}~\std{\tau_{R_1}}
                  \\
                  \std{\Gamma \vdash m_2(f_{0_2},\ldots,f_{n_2}) : }~\qual{q_{R_2}}~\std{\tau_{R_2}}
                  \\
                  \std{\Gamma \vdash} \qual{q_{R_1}}~\std{\tau_{R_1}~\sqsubseteq}~\qual{q_{R_2}}~\std{\tau_{R_2}}
                  \\
                  \std{\Gamma \vdash \forall i \in 0,\ldots,n_1 . ~f_{i_1} :~} \qual{q_{i_1}}~\std{\tau_{i_1}}
                  \\
                  \std{\Gamma \vdash \forall i \in 0,\ldots,n_2 . ~f_{i_2} :~} \qual{q_{i_2}}~\std{\tau_{i_2}}
                  \\
                  \std{\Gamma \vdash \forall i \in 0,\ldots,n_1 . ~} \qual{q_{i_1}}~\std{\tau_{i_1}~\sqsubseteq}~\qual{q_{i_2}}~\std{\tau_{i_2}}
                  \\
                  \std{\vdash n_1~=~n_2}
                  \\
                  \infr{\infEnv \vdash m_1(f_{0_1},\ldots,f_{n_1})~:~q_{R_1-I}~\tau_{R_1}}
                  \\
                  \infr{\infEnv \vdash m_2(f_{0_2},\ldots,f_{n_2})~:~q_{R_2-I}~\tau_{R_2}}
                  \\
                  \infr{\infEnv \vdash \forall i \in 0,\ldots,n_1 . ~f_{i_1}~:~q_{i_1-I}~\tau_{i_1}}
                  \\
                  \infr{\infEnv \vdash \forall i \in 0,\ldots,n_2 .~f_{i_2}~:~q_{i_2-I}~\tau_{i_2}}
                }
                {
                  \std{\Gamma \vdash m_1(f_{0_1},\ldots,f_{n_1})~\mathit{is~a~valid~override~of}~m_2(f_{0_2},\ldots,f_{n_2})}
                  \\
                  \infr{\infEnv \vdash m_2(f_{0_2},\ldots,f_{n_2})~:~\mathit{LUB_Q}(q_{R_1-I}, q_{R_2-I})~\tau_{R_2}}
                  \\
                  \infr{\infEnv \vdash \forall i \in 0,\ldots,n_2 . ~f_{i_2}~:~\mathit{LUB_Q}(q_{i_1-I},~q_{i_2-I})~\tau_{i_2} }
                }
                
  \end{mathpar}

  \todo{In $m(f_0,\ldots,f_n)$ I don't see types.  Does $f_i$ stand for $\|qual|
    \|basetype| \|varname|$?  Or maybe it stands for $q \tau$?}

  \todo{In ``$\infEnv \vdash x.f$'' in \textsc{field-assign}, this is about
    the declaration of $f$, not about the use with $x$ as receiver.  So it
    should be ``decl($f$)'' or maybe ``class.$f$'' or the like.}

  \todo{In \textsc{return}, its a bit surprising to see the subscript F in
    $\std{\Gamma \vdash m(f_0,\ldots,f_n)~:}~\qual{q_F}~\std{\tau_F}$.  Why
    is that subscript F and not R?  It could stand explanation.}

  \todo{These rules do not seem to handle qualifier polymorphism, where the
    return type can depend on the instantiation.  Somewhere the paper
    should discuss this, and whether our inference algorithm can handle it
    (maybe only when written by the programmer?)
    and which types of polymorphism our algorithm can handle and the
    challenges thereto.  This relevant to both \textsc{INVOKE} and \textsc{RETURN}.}

  \caption{Modified type rules used by our pluggable type framework. \std{Gray} indicates
    standard type rules for a Java-like language. \qual{Black} indicates additions to support
    pluggable typechecking. \infr{Red} indicates additions to support inference, \ie our
    contribution in this paper.
    Throughout, ``R'' subscripts refer to return types; ``F'' to formal parameters; ``A'' to
    actual arguments; and ``I'' to inference results.
    \todo{Which of the qualifier variables can be ``unqualified''?  I guess
      it is all of them?}
    In as assignment \<x=y>, \<x> is the ``formal'' and \<y> is the ``actual''.
    Type rules that do not require modification to support inference
    are elided for space.}
  \label{fig:type-rules}
\end{figure*}

The modified type rules appear in \cref{fig:type-rules}. \std{Gray} indicates
standard type rules for a Java-like language. \qual{Black} indicates additions to support
pluggable typechecking. \infr{Red} indicates additions to support inference, \ie our
contribution in this paper.
%
Type rules that do not need to be modified to support inference are elided for space.
%
To read these type rules, we first need to define some terms.

$\infEnv$ is the \emph{inference environment}, similar to the standard (qualified)
typing environment $\Gamma$. $\Gamma$ maps expressions and declarations to qualified types.
$\infEnv$ maps declarations to possibly-qualified types.
$\infEnv$ only maps declarations because our inference procedure does not need to infer
types for expressions: in fact, we assume that $\Gamma$ already does so (via flow-sensitive
refinement). Rather, $\infEnv$'s purpose is to map declarations to the results of inference.
Unlike $\Gamma$, the values in $\infEnv$ are \emph{possibly-qualified}, meaning that they can either
be qualified types or unqualified types. Initially, $\infEnv$ contains qualified types only
for declarations that were qualified before the current round of inference (which may come from
the programmer or from a previous inference round in the \cref{alg:wpi-fixpoint} fixpoint loop).
Once the current round of inference terminates, $\infEnv$ is used to produce the results of the
inference round by returning the set of all qualified types: any type that remains unqualified
throughout inference is not annotated, because no information about it was learned.

A pluggable type \emph{checker} permits programmers to leave basetypes unqualified.
On APIs (class/method/field declarations), the type checker uses defaulting
rules to assign a qualifier to each unqualified basetype.
Within a code block, the type checker performs local flow-sensitive type
inference.


We define the function $\mathit{LUB_Q}(q_1, q_2)$ to account for possibly-qualified types.
$q_1$ and $q_2$ are each either a type qualifier or ``not present''.
If both arguments are qualifiers, then the result of $\mathit{LUB_Q}$
is just their least upper bound. If only one qualifier is present, then $\mathit{LUB_Q}$'s result
is that qualifier; if both qualifiers are not present, $\mathit{LUB_Q}$'s result is ``not present'',
resulting in an unqualified result.

\todo{Then, this section should describe the interesting parts of the
  modified type rules in detail? At a minimum, we should probably walk the reader through
  one of the type rules.}


\todo{Here is a fact that I think we should explain to readers.
  Within a single type-checking run, the type estimates only go up.
  Between runs (that is, in \cref{alg:wpi-fixpoint}), the type estimates
  only go down.}


\todo{Somewhere, discuss completeness.  For a correct and verifiable
  program, this algorithm is complete for type qualifiers.  Also for pre-
  and post-conditions, I think.  But not for declaration annotations, for
  which every type system must write its own special-case code:  heuristics
  or other tweaks.  Most type systems don't have declaration annotations
  (no extra work or customizations for ``standard''\todo{poor word} type
  systems), but many do and need these enhancements.}

% LocalWords:  typechecker typechecked typechecking typecheckers intra
% LocalWords:  decl standard''


\section{Practical Considerations}
\label{sec:difficulties}

This section focuses on other things we had to do beyond
the core WPI algorithm presented in the previous section
in order to get a working tool. Its subsections should
be massaged to make them sound like theoretical problems---the idea
is that the reader should think of this section as the list of
problems we had to solve along the way, and the solutions to
the problems.

\subsection{Infinite Descending Chains}
\label{sec:infinite-descending-chains}

The usual lattice definition in an abstract interpretation
(or equivalently, a type system, cite Cousot 1997) forbids
infinite ascending chains but permits infinite descending chains.
The WPI fixpoint algorithm has a problem with them, though, and
needs widening operators. Talk about the early problems
with WPI on the Value Checker, where WPI would run for hundreds of
iterations: @IntRange(1, 10) -> @IntRange(1, 11) -> \ldots.

\subsection{Pre- and Post-conditions}
\label{sec:pre-post-conditions}

Discuss some of the troubles that Mike encountered when
he implemented pre- and post-condition support in WPI. Frame
this as a theoretical problem.

\subsection{Output Format}
\label{sec:output}

Find a way to frame the various WPI modes (i.e., JAIF mode,
stub mode, ajava mode) as a solution to a theoretical problem.

\todo{Think more about this and remember what other big problems
  we had to solve, and then add corresponding subsections.}


\section{Implementation}
\label{sec:implementation}

We built it and it works. Say things about the Checker Framework. This section
should be pretty short; it might give some more details about how we solved
some of the theoretical problems we describe in \cref{sec:difficulties}?


\section{Evaluation}
\label{sec:evaluation}

We aim to answer two research questions:
\begin{itemize}
\item \textbf{RQ1:} is our approach \emph{effective}: can infer most of the
  annotations that humans have written in prior case studies of pluggable typecheckers?
\item \textbf{RQ2:} is our approach \emph{general}: without any extra per-checker work,
  can we apply it to many extant pluggable typecheckers?.
\end{itemize}

%% This is the primary table.

\begin{table*}
  \caption{
    WPI performance on human-annotated benchmarks. ``Annotated for'' is
    the names of the typecheckers for which human annotations were written
    (see \cref{tab:checkers} to understand the codes).
    ``NCNB LoC'' is the number of lines of non-comment, non-blank code.
    ``Annos.'' is the number of human-written annotations, across all typecheckers.
    ``WPI \% Inferred'' is the percentage of the human-written annotations that WPI
    inferred.
  }
  \label{tab:case-studies}
  \posttablecaption

  \begin{tabular}{@{}ll|rrr@{}}
    \textbf{\smaller{Benchmark}} & Annoted for & NCNB LoC & Annos. & WPI \% Inferred \\
    \textbf{\smaller{JFreeChart}} & In & 95,000 & 2,500 & X\% \\
    \textbf{\smaller{PlumeUtil}} & All & 10,000 & ? & X\% \\
    \ldots & \ldots & \ldots & \ldots & \ldots \\
  \end{tabular}
\end{table*}


We answer both of these research questions with one set of
experiments.
%
Our methodology (at a high-level) is to collect a set of
projects that have been annotated by humans so that some pluggable
typechecker can typecheck them,
remove the human-written type annotations,
and then apply our approach.
%
To answer \textbf{RQ1}, we compare the output of our approach to the
ground-truth human-written type annotations (\cref{sec:results}}.
%
\textbf{RQ2} is answered incidentally: \todo{no} modifications to
the overall approach were required to handle any of the \todo{number} typecheckers
we considered; we further discuss our approach's performance for different typecheckers
in \cref{sec:by-checker}.

\subsection{Methodology}
\label{sec:methodology}

We collected subject programs from two sources:
\begin{itemize}
\item a GitHub search for projects using the Checker Framework, and
\item case studies in the evaluations of papers about pluggable typecheckers built
  on the Checker Framework.
\end{itemize}

\todo{The following describes the experimental evaluation I want to do. I doubt we'll
  have this done completely by the time the first ISSTA deadline arrives on 10 November,
  so we may need to rewrite this section to just say that we sampled projects by convenience
  that obviously used the CF, i.e., plume-lib.}

To find projects using the Checker Framework on GitHub, we searched GitHub for characteristic
markers of Checker Framework integration in build scripts. For example, one of the search
queries we used was for application of the Checker Framework's Gradle plugin; we used similar
searche queries for other common integrations and build systems, which we derived from
the Checker Framework manual. Our goal was to find projects that actively used a checker---and
consequently had annotations that could serve as ground truth for our experiments---so
searching for build system integration guarantees that the projects that we found at least
have (attempted to) run a checker at some time in the past. This stage of our search
resulted in \todo{X} projects. We then filtered those projects by removing those that
either did not build at all (\todo{Y} projects), contained no annotations (\todo{Z} projects),
or did not typecheck with the pluggable typechecker associated with their annotations (\todo{W}
projects), resulting in a dataset of \todo{V} projects with ground-truth annotations.
Call this dataset \textbf{DS-GH}.

We augmented this dataset by adding projects that had been hand-annotated for checkers
described in prior work. We checked each paper that has cited the original paper
about the Checker Framework~\cite{PapiACPE2008} to determine whether it includes any
case studies of Checker Framework checkers that are publicly available. \todo{X} papers
claimed such case studies; we could locate, build, and typecheck \todo{Y} projects (of
the \todo{Z} case studies claimed in these papers); call the resulting dataset \textbf{DS-PW}.

Together, \textbf{DS-GH} and \textbf{DS-PW} contain \todo{X} projects, \todo{Y} human-written
annotations that we use as ground truth, and \todo{Z} lines of non-comment, non-blank Java
source code.

For each project in \textbf{DS-GH} and \textbf{DS-PW}, we performed the following steps:
\begin{enumerate}
\item remove the existing, ground-truth annotations,
\item modify the build system to run the typechecker(s) using our tool,
\item write a short script to implement the ``outer loop'' of the algorithm
  (\ie \cref{alg:wpi-fixpoint}) by repeatedly invoking the project's build system, and
\item run this script to fixpoint to collect the final set of candidate type annotations
  inferred by our approach.
\end{enumerate}

Finally, we compare the final set of annotations generated by our approach to the original,
human-written ground-truth annotations. Our scripts and data are available
at~\todo{artifact location}.

\subsection{Results: Effectiveness}
\label{sec:results}

Our main results appear in \cref{tab:case-studies}. \todo{Discuss them.}

\subsection{Results: Generality}
\label{sec:by-checker}

%% This is the table per-checker.

\begin{table*}
  \caption{
    The results of \cref{tab:case-studies} indexed by typechecker
    rather than by benchmark. ``Code'' is the two-character code
    used in the ``Annotated for'' column of \cref{tab:case-studies}.
    A code in \cref{tab:case-studies} of ``All'' means that every
    checker in this table was run on that project.\todo{Do we need the
      ``All'' code?}
  }
  \label{tab:checkers}
  \posttablecaption

  \begin{tabular}{@{}ll|rrrr@{}}
    \textbf{\smaller{Checker}} & Code & \# projects & NCNB LoC & Annos. & WPI \% Inferred \\
    \textbf{\smaller{Index}} & In & 2 & 120,000 & 3,200 & X\% \\
    \textbf{\smaller{Nullness}} & Nu & 5 & 100,000 & ? & X\% \\
    \ldots & \ldots & \ldots & \ldots & \ldots \\
  \end{tabular}
\end{table*}


We also broke out the results by checker, in \cref{tab:checkers}. \todo{Discuss
  the results, especially if there are significant differences between checkers..}

\todo{Also emphasize the absolute number of checkers that WPI applies to.}


\section{Comparison to Other Tools}
\label{sec:comparison}

In this section, we compare our generic approach to
checker-specific inference tools. We probably compare
on our nullness benchmarks against the nullness-specific
tools. We might compare against CFI.

We hope that the results show that WPI is about as good as
the other approaches. The relative advantage of WPI is that
it doesn't require any custom code, unlike the others.

A table presents the results.


\section{Limitations and Future Work}
\label{sec:limits}

\todo{This section should discuss the threats to validity of our experiments.}

\todo{This section should discuss the limitations of our approach, especially
  that we cannot infer polymorphic types at all.}

\todo{This section should discuss future work, especially any ideas we have about
  inferring polymorphic types.}


\section{Related Work}
\label{sec:relatedwork}

\subsection{Type Inference For Pluggable Type Systems}
\label{sec:rw:type-inference-pluggable}

Talk about specific type inference approaches for pluggable type systems, including those we
compared against in \cref{sec:comparison}. Probably also CASCADE.

\subsection{Type Inference Approaches}
\label{sec:rw:type-inference}

Talk about general type inference approaches, classic and modern.

\subsection{Pluggable Types}
\label{sec:pluggable}

Talk about other approaches to making type systems easier to use?

\todo{Any other general category of things I'm missing here?}

The best citation for Werner's work on WPI is \cite{XiangLD2020}.

Houdini also repeatedly runs a verification tool, until fixed point \cite{FlanaganJL01,FlanaganL2001:Houdini}.


\section{Conclusion}

\todo{This section concludes.}

%% In future work, we plan to develop
%% improved inference techniques for lightweight ownership annotations.  Since
%% these annotations can be added anywhere without impacting soundness (they
%% are verified, not trusted), genetic
%% search
%% and machine-learning techniques could be used to introduce them, using the
%% warnings emitted by \tool as the fitness function.


\begin{acks}
\todo{Find the list of people who have contributed code to WPI.}
\end{acks}

%% The next two lines define the bibliography style to be used, and
%% the bibliography file.
\bibliographystyle{ACM-Reference-Format}
\balance
\bibliography{bib/bibstring-abbrev,bib/types,bib/dispatch,bib/ernst,bib/soft-eng,bib/invariants,bib/crossrefs,temp}

%%
%% If your work has an appendix, this is the place to put it.

\end{document}
\endinput
%%
%% End of file `sample-sigplan.tex'.

% LocalWords:  Kushigian Chandrakana Nandi LoC CCF
